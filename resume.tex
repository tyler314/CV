\documentclass{resume}

\begin{document}
\begin{center}
{\LARGE \bf Tyler J. Roberts} \\[1mm]
\footnotesize
(262) 388-4050 $|$
tjroberts314@gmail.com $|$
\href{https://www.linkedin.com/in/tyleroberts}{https://www.linkedin.com/in/tyleroberts} $|$
\href{https://github.com/tyler314}{https://github.com/tyler314}
%\href{http://tyler.engineering}{http://tyler.engineering}
\end{center}
\begin{flushleft}

{\textbf{\large Summary}} \\
{
\footnotesize
\tab Software engineer with 5+ years experience in  development and design. Most of my recent work has been focused on building backend systems, including but not limited to infrastructure, APIs and microservices, and client/server systems; including the development of the frontend client side. 
\\[4mm]
}
% \underline{Computer Skills} - Python, Golang, Java, Javascript, C/C++, Git workflow, Linux, Cloud Services. \\[5mm]
\longunderline{\textbf{\large Experience }}\\[2mm]

\textbf{Amazon.com, Inc.}, Seattle, WA \hfill Dec. 2019 - Present\\
\textbf{Software Development Engineer, Alexa Identity \& Personality}\\
{\footnotesize
	\ttab - Work on the Alexa Content Management System (ACMS) used by content creators to author Alexa responses. As well as \ttab the Adaptive Response Management System (ARMS), a real time service responsible for filtering and selecting content to be \ttab served to our customers, with a tps of over 10k  and a maximum latency of 200 ms. \\
	\ttab - Lead engineer in delivering adapted responses for Smart Home requests. Such features include sharing insights to customers \ttab who make a bad request, listing device capabilities of the requested device, and having Alexa say ``Good Morning / Night" \ttab when a smart light is turned on or off, depending on time of day. \\
	\ttab - Developed a Brief / Verbose Mode for ACMS. This allowed the authoring of responses targeted to customers who have their \ttab Alexa devices' brief mode setting turned on, as well as to those who have it turned off. Developed the setting on the \ttab frontend, worked with internal 3P teams to read the brief mode setting from Alexa customers and ingested it into ARMS, \ttab and developed filtering logic to serve the correct responses based on individual requests. \\
	\ttab - Migrated all of Alexa personality content from the old content management system, to ACMS/ARMS. This was a three \ttab month endeavor that required the development of bulk migration scripts, working closely with a select few engineers, PMs, \ttab and content creators across all supported locales. Had daily standups just for this project. \\
	\ttab - Led our org in hiring software development engineers by conducting over 50 interviews in Q1 2022. This allowed our org to \ttab get a higher  head count for engineers we were allowed to hire. I received our orgs quarterly award for this effort. \\
	%\ttab - Social chair for my team and sister teams, responsible for organizing quarterly events to get teammates to socialize outside \ttab of a work setting. I felt this was especially important during the post-pandemic world where many teammates don't see each \ttab other as often due to remote work. \\
	\ttab- \textit{Tools \& Technologies: AWS stack, Kotlin, Javascript, React Framework, Git.}\\[3mm]
}

\textbf{Cockroach Labs}, New York, NY \hfill May 2019 - Sep. 2019\\
\textbf{Software Engineer, Bulk I/O}\\
{\footnotesize
	\ttab - Apart of the Bulk I/O team, which is responsible for bulk operations in the database, including backup and restore, and \ttab schema changes in the database. \\
	\ttab - Fixed issue where the dump operation would include interleaved table statements inside
	of create table statements, causing \ttab dependency issues upon a restore. \\
	\ttab - Designed features for the backup \& restore process to include additional metadata about the cluster. Required a redesign of \ttab how the process worked, and reduced manual steps by the user. \\
	\ttab- \textit{Tools \& Technologies: Golang, SQL, Git/Github.}\\[3mm]
}

\textbf{Dataminr}, New York, NY \hfill September 2018 - February 2019\\
\textbf{Software Engineer, Content Search}\\
{\footnotesize
\ttab - Work on developing the tools used to search and filter through the Twitter pipeline and other media sources. \\
\ttab - JVM code base comprised of Scala and Java. Applications ran on AWS, and much of the day to day work involved use \\ \ttab with the AWS tech stack. Kinesis was used for processing the large data sources, where S3 was used to store the raw data.\\
\ttab - Migrated Pastebin data source from Kinesis to Kafka. Involved dependencies between several teams, and required \\ \ttab collaboration amongst them. \\
\ttab- \textit{Tools \& Technologies: Scala, Java, Maven, Docker, AWS, Google Guice, Git/Gitlab, IntelliJ, MacOS.}\\[3mm]
}

%\textbf{IBM Corporation}, Yorktown Heights, NY \hfill July 2017 - August 2018\\
%\textbf{Software Developer, Watson Health Cloud}\\
%{\footnotesize
%	\ttab- Developed Python application on another team that moved their locally stored data and user accounts to Bluemix,
%	\ttab IBM's cloud platform as a service. \\
%	\ttab- Designed and engineered a web application from the ground up using HTML, CSS, Angular, and Bootstrap. \\
%	\ttab Responsible for the front-end, including page navigation using an MVC design pattern. \\
%	%\ttab - Used Docker to rigorously test the deployment of the web application onto a server. \\
%	\ttab- \textit{Tools \& Technologies: Python, Docker, HTML, Angular, Javascript, PyCharm, MacOS.}\\[4mm]
%}

%\textbf{UW-Madison Plasma Physics Dept.}, Madison, WI \hfill Jan. 2016 - May 2017 \\
%\textbf{Scientific Programmer}\\[2mm]
%{\footnotesize
%	\ttab- Inherited former Ph.D. candidate's Python application, and enhanced it to communicate with additional instrumentation.
%	%\ttab added to the experiment. \\
%	\ttab- Wrote new code to parse binary data recorded from experiments, and store it to a database. \\
%	\ttab- Wrote a GUI using tkinter to improve productivity for the team as they used the application for their research. \\
%	\ttab- Collaborated extensively with scientists and professors to deliver a fully functional application for their research. \\
%	\ttab- Organized and taught Python tutorials for graduate students, professors, and scientists unfamiliar with Python and OOP. \\
%	\ttab- \textit{Tools \& Technologies: Python (2.7), C++,  Matlab, MacOS, Linux.}\\[5mm]
%}

%\textbf{Intel Corporation}, Hillsboro, OR \hfill May 2016 - Aug. 2016\\
%\textbf{Pre-Silicon Validation Engineering Intern}\\[2mm]
%{\footnotesize
%\ttab- Improved debug tool by creating my own checkers and algorithms that were used to validate the SoC architecture. My \ttab improvements were able to detect and isolate several bugs found within the design.\\
%\ttab- Developed Python modules in large code base for validation teams to share key architectural, test and debug knowledge. \\
%\ttab- Enhanced a validation tool by developing features that created easy debug for members of the design team. Managed to \ttab increase the productivity of the developers and validators as well as save time for the company. \\
%\ttab- Collaborated with several teams within DDG to determine the best way to provide feedback in the debugging process. \\
%\ttab- \textit{Tools \& Technologies: Python (3.4), Perl, SystemVerilog, OVM/UVM, Unix.}\\[3mm]
%}

%\textbf{Micron Technology}, Longmont, CO \hfill May 2015 - Aug. 2015\\
%\textbf{Product Validation Engineering Intern}\\[5mm]
%{\footnotesize
%\ttab- Wrote python program that manipulated Micron's script documentation to convert Sphinx docs to Pydoc docs. \\ \ttab Application allowed user to browse through documentation, and translated the pages in real time.\\
%\ttab- Tested solid state drives to ensure they performed correctly when given certain commands.\\
%\ttab- Worked with Micron's test automation platform, and used FIO, an I/O benchmarking tool, for testing the SSDs.\\
%\ttab- \textit{Tools \& Technologies: Python (2.7), Bash, Linux, Git, JIRA, Jenkins.}\\[5mm]
%}

\longunderline{\textbf{\large Education}} \\
{\bigsize
	B.S. Computer Engineering, Computer Science, \& Mathematics with Physics Certificate (minor). \\
	University of Wisconsin - Madison, May 2017 } \\[5mm]

%\textbf{UW-Madison Plasma Physics Dept.}, Madison, WI \hfill Jan. 2013 - Present \\
%\textbf{Mechanic's Assistant}\\
%{\footnotesize
%\ttab- Work directly under the supervision of the mechanical engineer of the department.\\
%\ttab- Assist with mechanical issues and restoration of the equipment, work consistently with heavy %machinery.\\[10mm]
%}




%\longunderline{\textbf{\large Projects}} \\[2mm]
%\textbf{Take Data 3 - UW-Madison Plasma Physics} \\%\hfill July - Aug. 2015 \\
%{\footnotesize
%	\ttab- Improved Ph.D. candidate's code so that it would communicate with additional instrumentation added to the experiment. \\
%	\ttab - Collaborated extensively with scientists and professors to deliver a fully functional tool for their research. \\
%	\ttab - Taught new graduate students Python and the code base so they could carry on my work where needed. 
%}
%\\[2mm]
%\textbf{Debugging Tool - Intel Corporation} \\%\hfill July 2016 \\
%{\footnotesize
	%\ttab- Improved upon a post-silicon debugging tool in pre-silicon by adding additional checkers and algorithms for signal detection.\\
	%\ttab- Caught several bugs in the SoC and helped the designers to fix the bugs faster and more effectively. \\
	%\ttab- Added improved debug hints and documentation for debugging tool to improve efficiency of debugging across teams. %\\[2mm]
%}
%\textbf{Validation Tool - Intel Corporation} \\
%{\footnotesize
%	\ttab- Developed Python modules in large code base for validation teams to share key architectural, test and debug knowledge. \\
%	\ttab- Collaborated with several teams within DDG to determine the best way to provide feedback in the debugging process. \\
%	\ttab -  Developed features that created easy debug and enabled members of the design team to become effective debuggers. \\[4mm]
%}

%\textbf{Python Pydoc Wrapper - Micron Technology} \\%\hfill July - Aug. 2015 \\
%{\footnotesize
%\ttab- Wrote python program that manipulated Micron's script documentation to convert Sphinx docs %to Pydoc docs.\\
%\ttab- Overide Pydoc methods, and implemented my own class, methods and algorithms to complete %the task.\\
%\ttab- Creates html pages of the documentation with links to other test documentations. \\[5mm]
%}

%	\textbf{GitHub}:
% \href{https://github.com/Tyler314}{https://github.com/Tyler314} \\[3mm]

%\longunderline{\textbf{\large Volunteer}} \\[2mm]
%\textbf{Greater University Tutoring Service}, Madison, WI \hfill Feb. 2015 - May 2015 \\

%\textbf{Tutor for Calculus \& Trigonometry} \\
%{\footnotesize
%\ttab- Tutored a group of 5 students once per week, with each tutoring session lasting 2 hours. \\
%\ttab- Prepared for class by reviewing lecture material, creating example problems, and preparing detailed notes. \\
%\ttab- Nominated as one of the top tutors in the program at the end of the semester. \\[2mm]
%}
%\textbf{Final Project - Intro to Microprocessors (ECE 353)} \hfill May 2015\\
%- Designed a turn based fighting game entirely in C, on a Texas Instruments microcontroller.\\
%- Utilized peripherals such as the GPIO pins, joystick, LCD screen, Nordic Wireless Radio, and the UART; all of which had to be configured in the C code.\\[3mm]

%\textbf{LinkedIn}: %\href{https://www.linkedin.com/in/tyleroberts}{https://www.linkedin.com/in/tyleroberts}



\end{flushleft}



























\end{document}
