\documentclass{resume}

\begin{document}
\begin{center}
{\LARGE \bf Tyler J. Roberts} \\[1mm]
\footnotesize
(262) 388-4050 $|$
tjroberts314@gmail.com \\[1mm]
\href{https://www.linkedin.com/in/tyleroberts}{https://www.linkedin.com/in/tyleroberts} $|$
\href{https://github.com/tyler314}{https://github.com/tyler314}
%\href{http://tyler.engineering}{http://tyler.engineering}
\end{center}
\begin{flushleft}

{\textbf{\large Summary}} \\
{
\scriptsize
\tab Software engineer with 6+ years experience in  development and design. My most recent work has been well-balanced between building back-end systems, and front-end  web applications. I'm seeking a role where I can leverage the experience I have to be a contributing member to a passionate team, that also requires me to learn new skill sets to succeed.
\\[4mm]
}
\longunderline{\textbf{\large Experience }}\\[2mm]

\normalsize{\bf Software Engineer \& Project Manager}\\
\footnotesize{\bf Copper $+$ Honey Salon $\cdot$ Freelance}\\
\footnotesize{Feb 2023 $-$ Present $\cdot$ Remote}\\[1mm]
{\scriptsize
	$-$ https://www.copperandhoneysalon.com\\
	$-$ I work as the sole Software Engineer on a small business' website. I work directly with the owners to understand their needs, and design software solutions to meet them.\\
	$-$ Created timelines and delivery checkpoints, conducted demos for the owners, and managed all parties involved including an independent UX Designer.\\
	$-$ Leveraged AWS tools such as Cloudfront and AWS Amplify for caching and simple CI/CD, respectively.\\
	$-$ Designed solution for the client so they can easily add and remove employees from the site, seamlessly working across all screen sizes.\\
	$-$ Delivered fully functional website from scratch, and continuing to add new features for the clients.\\
	$-$ \textit{Tools \& Technologies: AWS, Javascript, React.js, Material UI.}
}\\[3mm]

\normalsize{\bf Software Engineer}\\
\footnotesize{\bf Warner Bros. Discovery $\cdot$ Contract}\\
\footnotesize{Mar 2023 $-$ Dec 2023 $\cdot$ Remote}\\[1mm]
{\scriptsize
	$-$ I worked on the Customer Data Ingestion Team within MAX streaming. Team owns the back-end system microservices that sends HTTP requests from MAX app users. The system consumes user data, enriches, and stores it to be used by downstream users.\\
	$-$ Within my first month I became a key team contributor by quickly becoming apart of on-call rotations, contributing to code reviews, and pushing code to assist in the MAX launch in May.\\
	$-$ Led efforts of designing novel integration tests, and setting up new Grafana monitoring $+$ alerts for the team's new MAX services, requiring research, cross-team collaboration and team sign-offs.\\
	$-$ I became the lead engineer tasked with onboarding a new service from a re-org. Required collaboration with three away teams, to understand it and get the service running before my contract's end.\\
	$-$ General Operational Excellence. Served on-call shifts, performed load tests before big launches, contributed to documentation for the new services we were building.\\
	$-$ \textit{Tools \& Technologies: AWS, Typescript, Node.js, Express.js, Kafka, Grafana, Kubernetes $+$ Docker.}
}\\[3mm]

\normalsize{\bf Software Engineer}\\
\footnotesize{\bf Amazon.com $\cdot$ Full-Time}\\
\footnotesize{Dec 2019 $-$ Jan 2023 $\cdot$ Seattle, WA}\\[1mm]
{\scriptsize
	$-$ I worked on the Alexa Content Management System (ACMS) used by content creators to author Alexa responses. As well as the Adaptive Response Management System (ARMS), a real time service responsible for filtering and selecting content to be served to our customers, with a tps of over 10k  and a maximum latency of 200 ms. \\
	$-$ Lead engineer in delivering adapted responses for Smart Home requests. Such features include sharing insights to customers who make a bad request, listing device capabilities of the requested device, and having Alexa say ``Good Morning / Night" when a smart light is turned on or off, depending on time of day. \\
	$-$ Developed a Brief / Verbose Mode for ACMS. This allowed the authoring of responses targeted to customers who have their Alexa devices' brief mode setting turned on, as well as to those who have it turned off. Developed the setting on the front-end, worked with internal 3P teams to read the brief mode setting from Alexa customers and ingested it into ARMS, and developed filtering logic to serve the correct responses based on individual requests. \\
	$-$ Migrated all of Alexa personality content from the old content management system, to ACMS/ARMS. This was a three month endeavor that required the development of bulk migration scripts, working closely with a select few engineers, PMs, and content creators across all supported locales. Had daily standups just for this project. \\
	$-$ Led our org in hiring software development engineers by conducting over 50 interviews in Q1 2022. This allowed our org to get a higher  head count for engineers we were allowed to hire. I received our orgs quarterly award for this effort. \\
	$-$ Social chair for my org, responsible for organizing quarterly events to get teammates socializing outside of work. I felt this was especially important during the post-pandemic world where many teammates don't see each other as often due to remote work. \\
	$-$ \textit{Tools \& Technologies: AWS, Kotlin, Javascript, React.js, CI/CD Code Pipelines, Git.}
}\\[3mm]

\newpage
\normalsize{\bf Software Engineer}\\
\footnotesize{\bf Cockroach Labs $\cdot$ Full-Time}\\
\footnotesize{May 2019  $-$ Sep 2019 $\cdot$ New York, NY}\\[1mm]
\textbf{Cockroach Labs}\\
{\scriptsize
	$-$ Apart of the Bulk I/O team, which is responsible for bulk operations in the database, including backup and restore, and schema changes in the database. \\
	$-$ Fixed issue where the dump operation would include interleaved table statements inside of create table statements, causing dependency issues upon a restore. \\
	$-$ Designed features for the backup \& restore process to include additional metadata about the cluster. Required a redesign of how the process worked, and reduced manual steps by the user. \\
	$-$ \textit{Tools \& Technologies: Golang, SQL, Git/Github.}\\[3mm]
}

\normalsize{\bf Software Engineer}\\
\footnotesize{\bf Dataminr $\cdot$ Full-Time}\\
\footnotesize{Sep 2018  $-$ Feb 2019 $\cdot$ New York, NY}\\[1mm]
{\scriptsize
$-$ Work on developing the tools used to search and filter through the Twitter pipeline and other media sources. \\
$-$ JVM code base comprised of Scala and Java. Applications ran on AWS, and much of the day to day work involved use with the AWS tech stack. Kinesis was used for processing the large data sources, where S3 was used to store the raw data.\\
$-$ Migrated Pastebin data source from Kinesis to Kafka. Involved dependencies between several teams, and required collaboration amongst them. \\
$-$ \textit{Tools \& Technologies: Scala, Java, Maven, Docker, AWS, Google Guice, Git/Gitlab, IntelliJ, MacOS.}\\[3mm]
}
\normalsize{\bf Software Engineer}\\
\footnotesize{\bf IBM Corporation $\cdot$ Full-Time}\\
\footnotesize{Jul 2017  $-$ Aug 2019 $\cdot$ Yorktown Heights, NY}\\[1mm]
{\scriptsize
	$-$ Developed Python application on another team that moved their locally stored data and user accounts to Bluemix, IBM's cloud platform as a service. \\
	$-$ Designed and engineered a web application from the ground up using HTML, CSS, Angular, and Bootstrap. Responsible for the front-end, including page navigation using an MVC design pattern. \\
	$-$ Used Docker to rigorously test the deployment of the web application onto the servers. \\
	$-$ \textit{Tools \& Technologies: Python, Docker, HTML, Angular, Javascript, PyCharm, MacOS.}\\[4mm]
}

%\textbf{UW-Madison Plasma Physics Dept.}, Madison, WI \hfill Jan. 2016 - May 2017 \\
%\textbf{Scientific Programmer}\\[2mm]
%{\footnotesize
%	\ttab- Inherited former Ph.D. candidate's Python application, and enhanced it to communicate with additional instrumentation.
%	%\ttab added to the experiment. \\
%	\ttab- Wrote new code to parse binary data recorded from experiments, and store it to a database. \\
%	\ttab- Wrote a GUI using tkinter to improve productivity for the team as they used the application for their research. \\
%	\ttab- Collaborated extensively with scientists and professors to deliver a fully functional application for their research. \\
%	\ttab- Organized and taught Python tutorials for graduate students, professors, and scientists unfamiliar with Python and OOP. \\
%	\ttab- \textit{Tools \& Technologies: Python (2.7), C++,  Matlab, MacOS, Linux.}\\[5mm]
%}

%\textbf{Intel Corporation}, Hillsboro, OR \hfill May 2016 - Aug. 2016\\
%\textbf{Pre-Silicon Validation Engineering Intern}\\[2mm]
%{\footnotesize
%\ttab- Improved debug tool by creating my own checkers and algorithms that were used to validate the SoC architecture. My \ttab improvements were able to detect and isolate several bugs found within the design.\\
%\ttab- Developed Python modules in large code base for validation teams to share key architectural, test and debug knowledge. \\
%\ttab- Enhanced a validation tool by developing features that created easy debug for members of the design team. Managed to \ttab increase the productivity of the developers and validators as well as save time for the company. \\
%\ttab- Collaborated with several teams within DDG to determine the best way to provide feedback in the debugging process. \\
%\ttab- \textit{Tools \& Technologies: Python (3.4), Perl, SystemVerilog, OVM/UVM, Unix.}\\[3mm]
%}

%\textbf{Micron Technology}, Longmont, CO \hfill May 2015 - Aug. 2015\\
%\textbf{Product Validation Engineering Intern}\\[5mm]
%{\footnotesize
%\ttab- Wrote python program that manipulated Micron's script documentation to convert Sphinx docs to Pydoc docs. \\ \ttab Application allowed user to browse through documentation, and translated the pages in real time.\\
%\ttab- Tested solid state drives to ensure they performed correctly when given certain commands.\\
%\ttab- Worked with Micron's test automation platform, and used FIO, an I/O benchmarking tool, for testing the SSDs.\\
%\ttab- \textit{Tools \& Technologies: Python (2.7), Bash, Linux, Git, JIRA, Jenkins.}\\[5mm]
%}

\longunderline{\textbf{\normalsize Education}} \\
{\footnotesize
	B.S. Computer Engineering, Computer Science, \& Mathematics with Physics Certificate (minor). \\
	University of Wisconsin - Madison, May 2017 } \\[5mm]

%\textbf{UW-Madison Plasma Physics Dept.}, Madison, WI \hfill Jan. 2013 - Present \\
%\textbf{Mechanic's Assistant}\\
%{\footnotesize
%\ttab- Work directly under the supervision of the mechanical engineer of the department.\\
%\ttab- Assist with mechanical issues and restoration of the equipment, work consistently with heavy %machinery.\\[10mm]
%}




%\longunderline{\textbf{\large Projects}} \\[2mm]
%\textbf{Take Data 3 - UW-Madison Plasma Physics} \\%\hfill July - Aug. 2015 \\
%{\footnotesize
%	\ttab- Improved Ph.D. candidate's code so that it would communicate with additional instrumentation added to the experiment. \\
%	\ttab - Collaborated extensively with scientists and professors to deliver a fully functional tool for their research. \\
%	\ttab - Taught new graduate students Python and the code base so they could carry on my work where needed. 
%}
%\\[2mm]
%\textbf{Debugging Tool - Intel Corporation} \\%\hfill July 2016 \\
%{\footnotesize
	%\ttab- Improved upon a post-silicon debugging tool in pre-silicon by adding additional checkers and algorithms for signal detection.\\
	%\ttab- Caught several bugs in the SoC and helped the designers to fix the bugs faster and more effectively. \\
	%\ttab- Added improved debug hints and documentation for debugging tool to improve efficiency of debugging across teams. %\\[2mm]
%}
%\textbf{Validation Tool - Intel Corporation} \\
%{\footnotesize
%	\ttab- Developed Python modules in large code base for validation teams to share key architectural, test and debug knowledge. \\
%	\ttab- Collaborated with several teams within DDG to determine the best way to provide feedback in the debugging process. \\
%	\ttab -  Developed features that created easy debug and enabled members of the design team to become effective debuggers. \\[4mm]
%}

%\textbf{Python Pydoc Wrapper - Micron Technology} \\%\hfill July - Aug. 2015 \\
%{\footnotesize
%\ttab- Wrote python program that manipulated Micron's script documentation to convert Sphinx docs %to Pydoc docs.\\
%\ttab- Overide Pydoc methods, and implemented my own class, methods and algorithms to complete %the task.\\
%\ttab- Creates html pages of the documentation with links to other test documentations. \\[5mm]
%}

%	\textbf{GitHub}:
% \href{https://github.com/Tyler314}{https://github.com/Tyler314} \\[3mm]

%\longunderline{\textbf{\large Volunteer}} \\[2mm]
%\textbf{Greater University Tutoring Service}, Madison, WI \hfill Feb. 2015 - May 2015 \\

%\textbf{Tutor for Calculus \& Trigonometry} \\
%{\footnotesize
%\ttab- Tutored a group of 5 students once per week, with each tutoring session lasting 2 hours. \\
%\ttab- Prepared for class by reviewing lecture material, creating example problems, and preparing detailed notes. \\
%\ttab- Nominated as one of the top tutors in the program at the end of the semester. \\[2mm]
%}
%\textbf{Final Project - Intro to Microprocessors (ECE 353)} \hfill May 2015\\
%- Designed a turn based fighting game entirely in C, on a Texas Instruments microcontroller.\\
%- Utilized peripherals such as the GPIO pins, joystick, LCD screen, Nordic Wireless Radio, and the UART; all of which had to be configured in the C code.\\[3mm]

%\textbf{LinkedIn}: %\href{https://www.linkedin.com/in/tyleroberts}{https://www.linkedin.com/in/tyleroberts}



\end{flushleft}



























\end{document}
